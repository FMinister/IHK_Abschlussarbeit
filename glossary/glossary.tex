\newglossaryentry{Sammeldruckverfahren}
{
  name={Sammeldruckverfahren},
  description={, ein Produktionsverfahren, bei dem versucht wird, Aufträge verschiedener Kunden aber gleichem Papier zusammen auf einen Bogen zu platzieren. Die Minimierung der Weißfläche senkt die Produktionskosten und damit verbunden den Preis für den Kunden}
}

\newglossaryentry{REST}
{
  name={REST},
  description={, auch Representational State Transfer oder RESTful, ist ein Architekturstil, der die Kommunikation zwischen Maschinen vereinheitlicht und erleichtert}
}

\newglossaryentry{Docker Swarm}
{
  name={Docker Swarm},
  description={, ein Cluster aus Docker-Containern}
}

\newglossaryentry{RabbitMQ}
{
  name={RabbitMQ},
  description={, eine Open Source Message Broker Software, die das \acf{AMQP} implementiert. \autocite{rabbitmq}}
}

\newglossaryentry{Microservices}
{
  name={Microservices},
  description={, sind kleine, auf eine Reihe von Funktionen ausgelegte Services, die sich auf die Lösung eines bestimmten Problems konzentrieren. \autocite{aws_microservice}}
}

\newglossaryentry{Microservices-Architektur}
{
  name={Microservices-Architektur},
  description={, zielt darauf ab, Software als kleine, einzelne Einheiten aufzubauen, die unabhängig voneinander deployed und verwaltet werden können. \autocite{microservices_architekturen}}
}

\newglossaryentry{YOLOv7}
{
  name={YOLOv7},
  description={, auch \enquote{you only look once, version 7}, stellt als Programmbibliothek einen hochmodernen Objektdetektor zur Verfügung. \autocite{yolov7}}
}

\newglossaryentry{OpenCV}
{
  name={OpenCV},
  description={, stellt als Programmbibliothek Algorithmen für die Bildverarbeitung und Computer Vision zur Verfügung. \autocite{opencv}}
}

\newglossaryentry{API}
{
  name={API},
  description={, auch Application Programming Interface, ist eine Anwendungsschnittstelle, die von anderen Anwendungen angesprochen werden kann. \autocite{api}}
}

\newglossaryentry{Fully Convolutional Neural Network}
{
  name={Fully Convolutional Neural Network},
  description={, eine Architektur, die hauptsächlich für die \textit{\gls{Semantische Segmentierung}} verwendet wird. \autocite{fcnn}}
}

\newglossaryentry{Semantische Segmentierung}
{
  name={Semantische Segmentierung},
  description={,  ist ein \textit{\gls{Deep Learning}}-Algorithmus, der jedem Pixel in einem Bild eine Bezeichnung oder Kategorie zuordnet. \autocite{semantische_segmentierung}}
}

\newglossaryentry{Deep Learning}
{
  name={Deep Learning},
  description={, ein Teilgebiet des maschinellen Lernens, spezialisiert auf große Datenmengen.  \autocite{wuttke_deep}}
}

\newglossaryentry{Transfer Learning}
{
  name={Transfer Learning},
  description={, eine Methode von \textit{\gls{Deep Learning}}, bei dem der Lernfortschritt eines bestehenden Modells transferiert wird.  \autocite{wuttke_transfer}}
}

\newglossaryentry{Jittern}
{
  name={Jittern},
  description={, bezeichnet das zeitliche Zittern in der Übertragung von Signalen. \autocite{jittern}}
}

\newglossaryentry{Monolith}
{
  name={Monolith},
  description={, bezeichnet eine Softwarearchitektur, bei der eine Anwendung als eine einzelne und zusammenhängende Einheit gebaut wird. \autocite{monolith}}
}

\newglossaryentry{HA-Cluster}
{
  name={HA-Cluster},
  description={, Hochverfügbarkeitscluster, eine Anzahl von vernetzten Computern/Servern, die mit einem Minimum an Ausfallzeiten zuverlässig genutzt werden können. \autocite{cluster}}
}

\newglossaryentry{Verpackungsgesetz}
{
  name={Verpackungsgesetz},
  plural={Verpackungsgesetzes},
  description={, setzt seit 2019 die europäische Verpackungsrichtlinie 94/62/EG in Deutschland um und regelt das Inverkehrbringen von Verpackungen sowie die Rücknahme und Verwertung von Verpackungsabfällen. \autocite{verpackungsgesetz}}
}