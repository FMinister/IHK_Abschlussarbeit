\section{Dokumentation}

Für alle Services wurde eine Entwicklerdokumentation erstellt. Diese ist in \vref{appendix:sec:dokumentation} zu finden. Bei der Dokumentation wurde vor allem darauf Wert gelegt, Projekt- und Service-spezifische Punkte zu dokumentieren, sodass das Pflegen der bestehenden Anlage und Anwendungen sowie eine Replikation möglich ist. So wird die Einbindung der intern entwickelten \textit{\gls{RabbitMQ}}-Bibliothek nicht erklärt, da es dafür bereits eine Anleitung gibt. Auf das Trainieren eines neuen Modells oder den Aufbau des Sensorträgers wird dafür genauer eingegangen. Insgesamt soll die Dokumentation einen groben Überblick über die Funktionen und Herangehensweise der Services geben. Hyperlinks des internen Speichers, GitLab- und Confluence-Systems sind für diese Dokumentation angepasst, um personenbezogene und vertrauliche Daten zu schützen. Intern bekannte Abkürzungen werden nicht genauer erklärt.